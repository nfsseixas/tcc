\documentclass[12pt]{report}
%--------------------Pacotes-------------------------------
\usepackage[brazil]{babel}
\usepackage[utf8]{inputenc}
\usepackage{graphicx,color}
\usepackage{amsthm,amsfonts}
\usepackage{amssymb}
%---------------------Definições----------------------------
\setlength{\textwidth}{17 cm}
\setlength{\textheight}{20 cm}
\evensidemargin 0 cm
\oddsidemargin 0 cm
\setlength\parskip{4 pt}  
%------------------------------------------------------------
\title{Sistema de Frege}
\begin{document}
\maketitle
A primeira formulação da lógica proposicional como o cálculo formal se deve a G.Frege(1879). Frege usou o seguinte sistema de axiomas:\\
p$\rightarrow$(q$\rightarrow$p).\\
(s$\rightarrow$(p$\rightarrow$q))$\rightarrow$((s$\rightarrow$p)$\rightarrow$(s$\rightarrow$q)).\\
(p$\rightarrow$(q$\rightarrow$r))$\rightarrow$(q$\rightarrow$(p$\rightarrow$r))\\
(p$\rightarrow$q)$\rightarrow$($\neg$q$\rightarrow$$\neg$p).\\
Eukasiewicz simplificou reduzindo para 3 axiomas:
p$\rightarrow$(q$\rightarrow$p).\\
(s$\rightarrow$(p$\rightarrow$q))$\rightarrow$((s$\rightarrow$p)$\rightarrow$(s$\rightarrow$q)).\\
($\neg$p$\rightarrow$$\neg$q) $\rightarrow$(q$\rightarrow$p).\\
O nosso sistema é o seguinte:
(A1) ($\varphi$ $\rightarrow$ $\psi$ $\rightarrow$$\chi$) $\rightarrow$ ($\varphi$$\rightarrow$$\psi$)$\rightarrow$($\varphi$$\rightarrow$$\chi$).\\
(A2) $\varphi$ $\rightarrow$$\psi$ $\rightarrow$$\varphi$.\\
(A3)$\neg$$\varphi$$\rightarrow$$\varphi$ $\rightarrow$$\psi$.\\
(A4) ($\varphi$$\rightarrow$$\psi$)$\rightarrow$($\neg$$\varphi$$\rightarrow$$\psi$)$\rightarrow$$psi$.\\
Base de conectivos {$\rightarrow$,$\neg$}.\\
Regra de inferência: Modus ponens.\\
Obs: $\vdash$ é definido como sempre.
\begin{quote}
Lema 1: $\vdash$$\varphi$ $\rightarrow$ $\varphi$, isto é, $\varphi$ $\rightarrow$ $\varphi$ é um teorema.
\end{quote}.
\begin{proof}
($\varphi$($\varphi$ $\rightarrow$ $\varphi$) $\rightarrow$ $\varphi$) implica ( $\varphi$ $\rightarrow$ $\varphi$ $\rightarrow$$\varphi$)(A1) 1\\
$\varphi$ $\rightarrow$($\varphi$ $\rightarrow$ $\varphi$) $\rightarrow$ $\varphi$.(A2) 2\\
($\varphi$ $\rightarrow$ $\varphi$ $\rightarrow$ $\varphi$)$\rightarrow$($\varphi$ $\rightarrow$ $\varphi$), por M.P 1,2.   3\\
$\varphi$ $\rightarrow$ $\varphi$$\rightarrow$$\varphi$. (A2)  4\\
$\varphi$ $\rightarrow$ $\varphi$ M.P 3,4    5\\
\end{proof}
\begin{quote} Definição: \\
$\phi$ é consistente se existe $\varphi$ tal que $\phi$ $\nvdash$$\varphi$ \\
$\phi$ é inconsistente se existe $\phi$ $\vdash$ $\varphi$ para toda $\varphi$ \\
$\phi$ é maximalmente consistente se $\phi$ é consistente e $\phi$ $\cup$ \{$\varphi$\} é inconsistente para qualquer $\varphi$ $\in$ FORM – $\phi$.\\
Obs: I – $\phi$ é inconsistente $\leftrightarrow$ $\phi$ $\vdash$ $\bot$.\\
II – $\phi$ é consistente $\leftrightarrow$ $\phi$ $\nvdash$ $\bot$ onde $\bot$$:=$ $\neg$($\varphi$ $\rightarrow$ $\varphi$). Mais precisamente, $\bot$ $:=$$\neg$(p$_{0}$$\rightarrow$p$_{0}$).
\end{quote}
\begin{proof}
I: '' $\Longrightarrow$'' é trivial.\\
''$\Longleftarrow$''\\
Seja $\phi$ $\vdash$ $\bot$. Isto é, $\phi$ $\vdash$ $\neg$(p$_{0}$$\rightarrow$p$_{0}$). Por lema 1, $\phi$ $\vdash$ (p$_{0}$$\rightarrow$p$_{0}$).\\
Por (A3), $\phi$ $\vdash$ $\rightarrow$ (p$_{0}$$\rightarrow$p$_{0}$)$\rightarrow$ $\psi$.\\
Por M.P $ \frac {\neg( p_{0}\rightarrow p_{0})  \neg(p_{0}\rightarrow p_{0})\rightarrow(p_{0}\rightarrow p_{0}) \rightarrow \psi} {(p_{0}\rightarrow p_{0})\rightarrow \psi}$.\\
Por M.P $ \frac {(p_{0} \rightarrow p_{0})  (p_{0} \rightarrow p_{0}) \rightarrow \psi} {\psi} $.\\
\textbf{Exercício. Provar o II. É análogo!}.\\
\end{proof}
\begin{quote}
Lema 2: Teorema da dedução. $\phi$ $\cup$ \{ $\varphi$\} $\vdash$ $\psi$ $\Longrightarrow$ $\phi$ $\vdash$ $\varphi$ $\rightarrow$ $\psi$. \textbf{Exercício}
\end{quote}

\begin{quote}
Lema 3: I – $\phi$ $\vdash$ $\varphi$ $\leftrightarrow$ fizao $\cup$ \{ $\neg$$\varphi$\} $\vdash$ $\bot$.\\
II – $\phi$ $\vdash$ $\neg$$\varphi$ $\leftrightarrow$ $\phi$ $\cup$ \{$\varphi$\} $\vdash$ $\bot$\\.
\end{quote}

\begin{proof}
I: $\phi$ $\vdash$ $\varphi$ $\Longrightarrow$ $\phi$ $\cup$ \{$\neg$$\varphi$\} $\vdash$ $\varphi$.Também : $\phi$ $\cup$ \{$\neg$$\varphi$\} $\vdash$ $\neg$$\varphi$.\\
(A3): $\phi$$\cup$ \{$\neg$$\varphi$\}$\vdash$$\neg$$\varphi$$\rightarrow$$\varphi$$\rightarrow$$\bot$\\
2x M.P $\phi$ $\cup$ \{$\neg$$\varphi$\}$\vdash$$\bot$
$\phi$ $\cup$ \{$\neg$$\varphi$\} $\vdash$ $\bot$ $\Longrightarrow$ $\phi$ $\cup$ \{$\neg$$\varphi$\} $\vdash$ $\varphi$.(Obs.Anterior).\\
$\phi$$\vdash$$\neg$$\varphi$$\rightarrow$$\varphi$.(Teorema da dedução).\\
$\phi$$\vdash$ $\varphi$$\rightarrow$$\varphi$. (Lema 1)\\
$\phi$ $\vdash$ $\varphi$ (Por A4(com $\varphi$ $=$ $\psi$) e 2x M.P).\\
II - \textbf{exercício}.\\
\end{proof}

\begin{quote}
Lema 4:(Teorema de Lindenbaum). Um conjunto consistente tem uma extensão maximalmente consistente.
\end{quote}

\begin{proof}
Seja p$_{0}$,p$_{1}$,p$_{2}$,..., uma enumeração de FORM. Seja $\phi$ consistente. Construímos  uma cadeia fizao$=$ $\phi$$_{0}$, $\leq$$\phi$$_{1}$, $\leq$,...,$\phi$$_{n}$, $\leq$... tal que cada $\phi$$_{n}$ é consistente e $\phi$* $:=$ $\cup$$\phi$$_{n}$,n$\in$N é maximalmente consistente, $\phi$$\subseteq$$\phi$*.\\
$\phi$$_{0}$:= $\phi$. Se $\phi$$_{n}$ é definido para n$\in$N, então definimos: $\phi$$_{n+1}$:= $\phi$$_{n}$ $\cup$\{$\varphi$$_{n}$\}, se $\phi$$_{n}$ é consistente. $\phi$$_{n}$, c.c.\\
Obviamente, cada $\phi$$_{n}$ é consistente, n=0,1,... $\phi$* é maximalmente consistente.\\
$\phi$* é consistente se não for consistente, logo $\phi$* $\vdash$$\bot$. Pela finitude de $\vdash$ existe um subconjunto finito $\Delta$ $\subseteq$ $\phi$ tal que $\Delta$ $\vdash$$\bot$. Mas $\Delta$ $\subseteq$$\phi$$_{i}$ para um i$\in$N $\Longrightarrow$ $\varphi$$_{i}$ $\vdash$ $\bot$ contradição!\\
$[$$\Delta$ $=$ \{ $\psi$$_{1}$,...,$\psi$$_{n}$ \} $\Delta$ $\bot$ $\cup$ $\phi$ $_{n}$ $\Longrightarrow$ $psi$$_{1}$$\in$ $phi$$_{i1}$,$psi$$_{2}$$\in$ $phi$$_{i2}$,...,$psi$$_{m}$$\in$ $phi$$_{im}$ $\Longrightarrow$ $\Delta$ $\subseteq$ $\phi$$_{i}$, onde n$\in$N, i$=$ max \{i$_{i1}$,...,i$_{im}$\}.\\
$\phi$* é consistente. $\phi$ é maximalmente consistente.\\
Supondo que não, então $\phi$* $\cup$ \{$\varphi$\} é consistente para um $\varphi$ $\in$ FORM-$\phi$*. Seja $\varphi$=$\varphi$$_{n}$ na enumeração de fórmulas. Pela construção, $\phi$$_{n}$ $\cup$ \{$\varphi$$_{n}$\} é consistente. Logo, $\varphi$$_{n}$$\in$ $\phi$$_{n+1}$ $\subseteq$$\phi$* é contradição.
\end{proof}
	
\end{document}
