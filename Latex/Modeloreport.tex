\documentclass[12pt,twocolumn]{report}
\usepackage[brazil]{babel}
\usepackage[utf8]{inputenc}
\usepackage{graphicx,color}
\usepackage{amsthm,amsfonts}
\setlength{\textwidth}{17 cm}
\setlength{\textheight}{20 cm}
\evensidemargin 0 cm
\oddsidemargin 0 cm
\setlength\parskip{4 pt} 
\title{título}


\begin{document}
\maketitle{}
\tableofcontents
 Texto e comandos de efeito local\\
fodam-se\\
%Aspas s~o digitadas assim:
‘‘Texto entre aspas’’.\\
%Texto em negrito deve s\er digitado como:
\textbf{Isto está em negrito}.\\
{\small pequeno é de tua mãe}
{\color{blue} this is sparta}
\begin{quote}
Texto a ser indentado.
\end{quote}
\begin{itemize}
\item Os itens  precedidos por $\bullet$;
\item Os itens  separados por um espaço adicional.
\end{itemize}

\indent A teoria de Ausubel prioriza a forma com que aprendemos e a maneira com quais
organizamos novas informações dentro do cérebro. Ele dá nome a essa organização como Estrutura
cognitiva. Ele define estrutura cognitiva como todo o conteúdo informacional presente em um
indivíduo, organizado em qualquer modalidade do conhecimento\footnote{sdhsdhasteste de rodapé}.

\indent Um ponto importante na sua teoria é como iremos associar os novos conhecimentos
aos pontos de âncora, ou seja, como iremos associar os novos conhecimentos aos que já temos. O
conhecimentos prévios serão de uma grande importância e influenciará a maneira como iremos
organizar cognitivamente as novas informações.

\end{document}
