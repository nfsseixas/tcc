\documentclass[12pt]{report}
%--------------------Pacotes-------------------------------
\usepackage[brazil]{babel}
\usepackage[utf8]{inputenc}
\usepackage{graphicx,color}
\usepackage{amsthm,amsfonts}
\usepackage{amssymb}
%---------------------Definições----------------------------
\setlength{\textwidth}{17 cm}
\setlength{\textheight}{20 cm}
\evensidemargin 0 cm
\oddsidemargin 0 cm
%\setlength\parskip{3 pt} 
\linespread{1.0} 
%------------------------------------------------------------
\title{Onisciência Lógica.}
\author{Nilton Flávio Sousa Seixas.}
\begin{document}
\maketitle
\tableofcontents
\chapter{Lógica Modal.}
%%%%%%%%%%%%%%%%%%%%%%%%%%%%%%%%%%%%%%%%%%%%%%%%%%%%%%%%%%%%%%%%%%%%%%%%%%%%%%%%%%%%%%%%%%%%%%%%%%%%%%%%%%%%%%%%%%%%%%%%%%%%%%%%%%
%%%%%%%%%%%%%%%%%%%%%%%%%%%%%%%%%%%%%%%%%%%%%%%%%%%%%%%%%%%%%%%%%%%%%%%%%%%%%%%%%%%%%%%%%%%%%%%%%%%%%%%%%%%%%%%%%%%%%%%%%%%%%%%%%%
\section{Sintaxe.}
    \hspace{0.5cm} Agora vou introduzir a sintaxe usada pela lógica modal proposicional. Irei fazê-lo em dois passos: o primeiro é a definição do alfabeto modal e depois, a linguagem modal induzida pelo alfabeto modal sobre o conjunto P.\\
              \indent Seja P um conjunto não-vazio de fórmulas atômicas que pode ser  finito para um n qualquer tal que P = \{p$_{1}$,p$_{2}$,p$_{3}$,...,p$_{n}$\} ou infinito tal que P = \{ p$_{i}$,p$_{i+1}$,...\}, para i $\geq$ 1, sendo que o mais importante é que os membros de P possam ser enumerados. Então temos como sintaxe da lógica modal:\\
\indent (1) Qualquer que seja p$_{i}$, tal que p$_{i}$ $\in$ P, para  i $\geq$ 1.\\
\indent (2) O simbolo da contradição $\perp$.\\
\indent (3) Operadores lógicos binários : $\rightarrow$,$\vee$,$\land$, sendo implicação, disjunção e conjunção respectivamente.\\
\indent (4) Os operadores modais $\Box$(necessidade) e $\Diamond$(possibilidade).\\
\indent (5) Os símbolos auxiliares (,).\\
\indent (6) Operador unário de negação $\neg$.\\
             \indent Agora apresento a definição da linguagem modal, induzida pelo alfabeto modal sobre o conjunto de fórmulas atômicas P. Seja FormM o cojunto mínimo de fórmulas da lógica modal, FormM satisfaz as seguintes propriedades : \\
\indent (7) p$_{i}$ $\in$ FormM para qualquer pi $\in$ P,  i $\geq$ 1.\\
\indent (8) $\perp$ $\in$ FormM.\\
\indent (9) Se $\varphi$,$\varphi$' $\in$ FormM, então as fórmulas: ($\varphi$ $\rightarrow$ $\varphi$'), ($\varphi$ $\land$ $\varphi$'), ($\varphi$ $\vee$ $\varphi$), $\Box$$\varphi$, $\Diamond$$\varphi$ $\in$ FormM.\\
\indent (10) Se $\varphi$,$\varphi$' $\in$ FormM, então as fórrmulas ($\Box$$\varphi$), ($\Diamond$$\varphi$),($\neg \varphi$) $\in$ FormM.\\
%%%%%%%%%%%%%%%%%%%%%%%%%%%%%%%%%%%%%%%%%%%%%%%%%%%%%%%%%%%%%%%%%%%%%%%%%%%%%%%%%%%%%%%%%%%%%%%%%%%%%%%%%%%%%%%%%%%%%%%%%%%%%%%%%
%%%%%%%%%%%%%%%%%%%%%%%%%%%%%%%%%%%%%%Semantica%%%%%%%%%%%%%%%%%%%%%%%%%%%%%%%%%%%%%%%%%%%%%%%%%%%%%%%%%%%%%%%%%%%%%%%%%%%%%%%%%%
\section{Semântica.}
 \hspace{0.5cm} Para representar a semântica da lógica modal é necessário representar o modelo de Kripke. Mas para isso, preciso introduzir noções de enquadramento.\\
 \indent Um enquadramento E é representado por uma tupla E = (S,R), onde:\\
 \indent S é um conjunto não-vazio de estados.\\
 \indent R é uma relação binária entre os estados. Quando um estado s se relaciona com outro estado s', denotamos a relação como R(s,s').\\
 \indent Seja V uma função que valora as variáveis nos estados, ela é denotada por:\\
 \indent V: S $\rightarrow $ (P $\rightarrow $  \{1,0\}), onde P é o conjunto das fórmulas atômicas, 1 representa a valoração verdade, e 0 a valoração falsa. S é o conjunto de estados do enquadramento.\\
 \indent Um modelo de Kripke é denotado por M = (E,V), onde E é um enquadramento e V uma função que atribui 1 ou 0 para as fórmulas nos estados. Um modelo de Kripke nada mais é do que uma estrutura de interpretação para fórmulas da lógica modal. Um mundo de kripke w, é a composição de um modelo de kripke M, com um estado s $\in$ S e S $\subseteq$ M. Basicamente é o mesmo que especificar um único estado de um modelo. Um estado, a depender da lógica modal, pode ser interpretado como um mundo, ou situação. Aqui eu tratarei como estado.
 %%%%%%%%%%%%%%%%%%%%%%%%%%%%%%%%%%%%%%%%%%%%%%%%%%%%%%%%%%%%%%%%%%%%%%%%%%%%%%%%%%%%%%%%%%%%%%%%%%%%%%%%%%%%%%%%%%%%%%%%%%%%%%%%%%%%
 \subsection{Satisfatibilidade em um estado.}%%%%%%%%%%%%%%%%%%%%%%%%%%%%%%%%%%%%%%%%%%%%%%%%%%%%%%%%%%%%%%%%%%%%%%%%%%%%%%%%%%%%%%%%%
 \hspace{0.5cm} Seja M um modelo de Kripe tal que M = (E,V) e s $\in$ S, S $\subseteq$ E. A satisfatibilidade de uma formula $\varphi$, $\varphi \in$ FormM, é denotada por :
 \begin{quote}
 \indent M,s $\vDash$ $\varphi$
 \end{quote}
 e segui-se indutivamente as definições:\\
\indent (1) M,s $\vDash$ p :$\Leftrightarrow$ V(s)(p) = 1 para p $\in$ P, ou seja, Uma fórmula é satisfeita em um estado se somente se, a valoração da fórmula naquele estado é 1.\\
\indent (2) M,s $\vDash$ $\varphi$ $\land$ $\varphi$' :$\Leftrightarrow$ M,s $\vDash$ $\varphi$ e M,s $\vDash$ $\varphi$'. A valoração de $\varphi$ em s é igual a 1, bem como a valoração de $\varphi$'.\\
\indent (3) M,s $\vDash$ $\varphi$ $\vee$ $\varphi$' :$\Leftrightarrow$ M,s $\vDash$ $\varphi$ ou M,s $\vDash$ $\varphi$'.\\
\indent (4) M,s $\vDash$ $\varphi$ $\rightarrow$ $\varphi$' :$\Leftrightarrow$ M,s $\nvDash$  $\varphi$ e M,s $\vDash$ $\varphi$' \footnote{definição alternativa: se M,s $\vDash$ $\varphi$, então M,s $\vDash$ $\varphi$'}. \\
\indent (5) M,s $\vDash$ $\neg$ $\varphi$ :$\Leftrightarrow$ M,s $\nvDash$ $\varphi$. Essa definição nos faz pensar no fato: M,s $\models \varphi \Leftrightarrow$  M,s $\nvDash \varphi'$, que é verdade, porém tem que ser demonstrado antes de fazer parte de uma prova.\\
\indent (6) M,s $\vDash$ $\Box$ $\varphi$ se para cada t $\in$ S tal que 
R(s,t), M,t $\vDash$ $\varphi$ . Veja que não necessariamente M,s $\models \varphi$ para que M,s $\models \square  \varphi$ \\
\indent (7) M,s $\vDash$ $\Diamond$$\varphi$ se existe t, tal que M,t $\vDash$ $\varphi$ e R(s,t)\footnote{Quando digo R(s,t), afirmo que há uma relação de acessibilidade do estado s ao estado t considerado pelo agente em questão.}.\\
%%%%%%%%%%%%%%%%%%%%%%%%%%%%%%%%%%%%%%%%%%%%%%%%%%%%%%%%%%%%%%%%%%%%%%%%%%%%%%%%%%%%%%%%%%%%%%%%%%%%%%%%%%%%%%%%%%%%%%%%%%%%%%%%%%%%%%
\subsection{Satisfatibilidade em um modelo de Kripke.}%%%%%%%%%%%%%%%%%%%%%%%%%%%%%%%%%%%%%%%%%%%%%%%%%%%%%%%%%%%%%%%%%%%%%%%%%%%%%%%%
\hspace{0.5cm} Seja M, um modelo de Kripke tal que M = (E,V). M satisfaz uma fórmula $\varphi$, $\varphi \in$ FormM, denotada por :
 \begin{quote}M $\vDash$ $\varphi$ 
 \end{quote}
 se M,s $\vDash$ $\varphi$ para todo s $\in$ S, S $\subseteq$ E, ou seja, $\varphi$ tem valoração 1 em todos os estados daquele modelo de Kripke. Dizemos que M é um modelo para $\varphi$.\\
 %%%%%%%%%%%%%%%%%%%%%%%%%%%%%%%%%%%%%%%%%%%%%%%%%%%%%%%%%%%%%%%%%%%%%%%%%%%%%%%%%%%%%%%%%%%%%%%%%%%%%%%%%%%%%%%%%%%%%%%%%%%%%%%%%%%%%
 \subsection{Exemplo para satisfatibilidade em um estado.}%%%%%%%%%%%%%%%%%%%%%%%%%%%%%%%%%%%%%%%%%%%%%%%%%%%%%%%%%%%%%%%%%%%%%%%%%%%%
\hspace{0.5cm} Seja as fórmulas atômicas \{x,y,z\} $\subseteq $ P. Considere um modelo de Kripke M tal que:\\
S = s$_{1}$,s$_{2}$,s$_{3}$,s$_{4}$,s$_{5}$.\\
R é uma relação tal que 1 $\leq $ i,j $\leq$ 5, s$_{i}$Rs$_{j}$ se j = i+1.\\
V(s$_{2}$)(x) = 1. \\
V(s$_{3}$)(x) = 1. \\
Para todo s$_{i}\in$ S, $\pi$(s$_{i}$)(y) = v, para i = 1,2,3,4,5.\\
V(s$_{2}$)($\neg$z) = 1.\\
Sejam as relações entre os estados:\\
s$_{1}$ R s$_{2}$\\
s$_{2}$ R s$_{3}$\\
s$_{3}$ R s$_{4}$\\
s$_{4}$ R s$_{5}$\\    
segue-se que:\\
(1) M,s$_{2}$ $\models$ x $\wedge$ ($\neg$z), pois M,s$_{2}$ $\models$ x, e M,s$_{2}$ $\models$ ($\neg$z).\\
(2) M,s$_{1}$ $\models$ $\square$ x, pois M,s$_{2}$ $\models$ x, e ele é o único estado que se relaciona com s$_{1}$, logo a afirmação procede.\\
(3) M,s$_{2}$ $\models$ $\square$ x. É similar ao anterior, porém no estado s$_{2}$, o estado possível considerado por um agente é o s$_{3}$. Tornando a única relação apartir de s$_{2}$, seguindo a definição 6 da seção 1.3.\\
(4) M,s$_{3}$ $\nvDash$ $\square$ x, pois M,s$_{4}$ $\nvDash$ x.\\
(5) M,s$_{1}$ $\nvDash$ ($\square$ x) $\rightarrow$ x, pois M,s$_{1}$ $\models$ $\square$ x porém M,s$_{1}$ $\nvDash$ x.\\
(6) M,s$_{1} \models   \Diamond (\square$ x) pois M,$s_{2} \models (\square$ x) e pela definição 7 da seção 1.3, basta que exista um estado que satisfaça a fórmula x.\\
(7) M,s$_{1}\Diamond$ (x $\wedge (\neg$ z)) pois M,s$_{2} \models$ x e M,s$_{2} \models \neg$ z.\\
(8) M,s$_{5} \nvDash \Diamond$ ($\neg$z). Teria que haver um estado s $\in $ S tal que s$_{5}$Rs ou seja, uma relação R(s$_{5}$,s).\\
(9) M,$s_{i}\models$ y. Pois para todo s$_{i}\in$ S, V(s$_{i}$)(y) = v, para i = 1,2,3,4,5.

%%%%%%%%%%%%%%%%%%%%%%%%%%%%%%%%%%%%%%%%%%%%%%%%%%%%%%%%%%%%%%%%%%%%%%%%%%%%%%%%%%%%%%%%%%%%%%%%%%%%%%%%%%%%%%%%%%%%%%%%%%%%%%%%%%%%%
 \subsection{Classes de enquadramentos}%%%%%%%%%%%%%%%%%%%%%%%%%%%%%%%%%%%%%%%%%%%%%%%%%%%%%%%%%%%%%%%%%%%%%%%%%%%%%%%%%%%%%%%%%%%%%%
 \hspace{0.5cm} Seja $\varepsilon$ a classe de todos os enquadramentos. Temos as seguintes subclasses de enquadramento:\\
 $\varepsilon _{reflexivo}$ é a classe dos enquadramentos reflexivos. Ou seja, $\forall$ s $\in$ S, temos R(s,s).\\
 $\varepsilon _{transitivo}$ é a classe dos enquadramentos transitivos. $\forall$ s, t ,u $\in$ S, se  R(s,t) e R(t,u), então R(s,u).\\
 $\varepsilon _{simetrico}$ é a classe dos enquadramentos simétricos. $\forall$ s, t, $\in$ S, se  R(s,t) então R(t,s).\\
 $\varepsilon _{euclidiano}$ é a classe dos enquadramentos euclidianos. $\forall$ s, t, u, $\in$ S, se  R(s,t) e R(s,u), então R(t,u).\\
 $\varepsilon _{serial}$ é a classe dos enquadramentos seriais. $\forall$ s, s $\in$ S, $\exists$ t, t $\in$ S, tal que R(s,t).\\
 %%%%%%%%%%%%%%%%%%%%%%%%%%%%%%%%%%%%%%%%%%%%%%%%%%%%%%%%%%%%%%%%%%%%%%%%%%%%%%%%%%%%%%%%%%%%%%%%%%%%%%%%%%%%%%%%%%%%%%%%%%%%%%%%%%
\subsection{Classe de modelos de kripke.}
 \hspace{0.5cm} Defino $\Theta$ como a representação da classe de modelos de kripke, ou seja, a classe de interpretações para enquadramentos E.
 %%%%%%%%%%%%%%%%%%%%%%%%%%%%%%%%%%%%%%%%%%%%%%%%%%%%%%%%%%%%%%%%%%%%%%%%%%%%%%%%%%%%%%%%%%%%%%%%%%%%%%%%%%%%%%%%%%%%%%%%%%%%%%%%%%%%

 \subsection{Validade em um estado.}%%%%%%%%%%%%%%%%%%%%%%%%%%%%%%%%%%%%%%%%%%%%%%%%%%%%%%%%%%%%%%%%%%%%%%%%%%%%%%%%%%%%%%%%%%%%%%%%%
 \hspace{0.5cm} Seja M um modelo de Kripke, E um enquadramento modal, $\varphi$ uma fórmula tal que $\varphi \in$ FormM. $\varphi$ é válida em um estado s, tal que s $\in$ S, S $\subseteq$ E, denotada por 
 \begin{quote}
 E,s $\models \varphi$,
 \end{quote}
 se M,s $\models \varphi$ para todos os modelos de Kripke M, tal que o enquadramento permaneça constante, ou seja, qualquer modelo de kripke baseado neste enquadramento E.
 %%%%%%%%%%%%%%%%%%%%%%%%%%%%%%%%%%%%%%%%%%%%%%%%%%%%%%%%%%%%%%%%%%%%%%%%%%%%%%%%%%%%%%%%%%%%%%%%%%%%%%%%%%%%%%%%%%%%%%%%%%%%%%%%%%%%
 \subsection{Exemplo para validade em um estado.}%%%%%%%%%%%%%%%%%%%%%%%%%%%%%%%%%%%%%%%%%%%%%%%%%%%%%%%%%%%%%%%%%%%%%%%%%%%%%%%%%%%%
 \hspace{0.5cm} Seja o enquadramento modal fixo E, uma fórmula $\varphi$, $\varphi \in $ FormM, que possui as seguintes relações e estados:\\
 S = \{s$_{1}$, s$_{2}$, s$_{3}$, s$_{4}$\}.\\
 s$_{1}$Rs$_{2}$.\\
 s$_{2}$Rs$_{3}$.\\
 s$_{3}$Rs$_{4}$.\\
 s$_{4}$Rs$_{1}$.\\
 Sejam as valorações V$_{1}$:\\
 V$_{1}$(s$_{1}$)($\varphi$) = 0;\\
 V$_{1}$(s$_{2}$)($\varphi$) = 1;\\
 V$_{1}$(s$_{3}$)($\varphi$) = 0;\\
 V$_{1}$(s$_{4}$)($\varphi$) = 0;\\
 Seja M$_{1}$ um modelo de Kripe tal que M$_{1}$ = (E,V$_{1}$). Sejam as valorações V$_{2}$:\\
 V$_{2}$(s$_{1}$)($\varphi$) = 1;\\
 V$_{2}$(s$_{2}$)($\varphi$) = 1;\\
 V$_{2}$(s$_{3}$)($\varphi$) = 0;\\
 V$_{2}$(s$_{4}$)($\varphi$) = 0;\\
 Seja M$_{2}$ um modelo de Kripe tal que M$_{2}$ = (E,V$_{2}$). Sejam as valorações V$_{3}$:\\
 V$_{3}$(s$_{1}$)($\varphi$) = 0;\\
 V$_{3}$(s$_{2}$)($\varphi$) = 1;\\
 V$_{3}$(s$_{3}$)($\varphi$) = 1;\\
 V$_{3}$(s$_{4}$)($\varphi$) = 0;\\
 Seja M$_{3}$ um modelo de Kripe tal que M$_{3}$ = (E,V$_{3}$). Seja $\Theta$ a classe dos modelos de Kripke e suponha que $ \Theta$ = \{M$_{1}$,M$_{2}$,M$_{3}$\}. Observe que para todos os modelos de Kripke baseados em E, M$_{x}$,s$_{2} \models \varphi$, para x = 1,2,3.
 %%%%%%%%%%%%%%%%%%%%%%%%%%%%%%%%%%%%%%%%%%%%%%%%%%%%%%%%%%%%%%%%%%%%%%%%%%%%%%%%%%%%%%%%%%%%%%%%%%%%%%%%%%%%%%%%%%%%%%%%%%%%%%%%%%%%
 \subsection{Validade de uma fórmula.}%%%%%%%%%%%%%%%%%%%%%%%%%%%%%%%%%%%%%%%%%%%%%%%%%%%%%%%%%%%%%%%%%%%%%%%%%%%%%%%%%%%%%%%%%%%%%%%
 \hspace{0.5cm} Seja uma fórmula $\varphi, \varphi \in$ FormM. Uma fórmula é valida, sendo denotada por
 \begin{quote}
 $\vDash$ $\varphi$ ($\varphi$ é válido) 
 \end{quote}
 se M $\vDash$ $\varphi$ para todo M.\footnote{Para todo os modelos de Kripke.}\\ 
 %%%%%%%%%%%%%%%%%%%%%%%%%%%%%%%%%%%%%%%%%%%%%%%%%%%%%%%%%%%%%%%%%%%%%%%%%%%%%%%%%%%%%%%%%%%%%%%%%%%%%%%%%%%%%%%%%%%%%%%%%%%%%%%%%%%%
 \subsection{ Consequência lógica Local.}%%%%%%%%%%%%%%%%%%%%%%%%%%%%%%%%%%%%%%%%%%%%%%%%%%%%%%%%%%%%%%%%%%%%%%%%%%%%%%%%%%%%%%%%%%%%
 \hspace{0.5cm} Sejam $\Psi$ um conjunto de fórmulas tal que $\Psi \subseteq $ FormM , $\varphi$, tal que $\varphi \in$ FormM, M um modelo de Kripke dado por M = (E,V), $\Vdash $ o símbolo da consequência lógica, 
 \begin{quote}
 $\varphi$ é consequência lógica local de $\Psi$, denotado por $\Psi \Vdash_{l} \varphi $, onde $_{l}$ representa local,
 \end{quote}
 se para todo s, tal que s $\in$ S, se M,s $\models \Psi $, então M,s $\models \varphi $. Essa definição atende apenas para um único modelo de Kripke.
 %%%%%%%%%%%%%%%%%%%%%%%%%%%%%%%%%%%%%%%%%%%%%%%%%%%%%%%%%%%%%%%%%%%%%%%%%%%%%%%%%%%%%%%%%%%%%%%%%%%%%%%%%%%%%%%%%%%%%%%%%%%%%%%%%%%%
 \subsection{Consequência lógica Global.}%%%%%%%%%%%%%%%%%%%%%%%%%%%%%%%%%%%%%%%%%%%%%%%%%%%%%%%%%%%%%%%%%%%%%%%%%%%%%%%%%%%%%%%%%%%%
 \hspace{0.5cm} Seguindos as mesmas configurações da seção anterior para M,$\varphi , \Psi $ e s:
 \begin{quote}
  \hspace{0.5cm} $\varphi$ é consequência lógica global de $\Psi$, denotado por $\Psi \Vdash \varphi $,
  \end{quote}
  se para todo modelo de Kripke M, se M $\models \Psi$ então M $\models \varphi $. 
  %%%%%%%%%%%%%%%%%%%%%%%%%%%%%%%%%%%%%%%%%%%%%%%%%%%%%%%%%%%%%%%%%%%%%%%%%%%%%%%%%%%%%%%%%%%%%%%%%%%%%%%%%%%%%%%%%%%%%%%%%%%%%%%%%%%
  \subsection{ Exemplo para consequência lógica global.}%%%%%%%%%%%%%%%%%%%%%%%%%%%%%%%%%%%%%%%%%%%%%%%%%%%%%%%%%%%%%%%%%%%%%%%%%%%%%
   \begin{quote}
   $\varphi \Vdash \square \varphi $.
   \end{quote}
   Prova. Por hipótese temos que:\\
   Qualquer que seja M, M $\models \varphi $ implica que M $\models \square \varphi $;\\
   Isso equivale a : se M,s $\models \Psi$ então M,s $\models \square \varphi$. Para todo s, s $\subseteq $ S, S $\subseteq $ M;\\
   Logo para todo t, tal que R(s,t), t $\subseteq $ S, M,t $\models \varphi $.\\
   Concluindo, M $\models \square \varphi $.
   %%%%%%%%%%%%%%%%%%%%%%%%%%%%%%%%%%%%%%%%%%%%%%%%%%%%%%%%%%%%%%%%%%%%%%%%%%%%%%%%%%%%%%%%%%%%%%%%%%%%%%%%%%%%%%%%%%%%%%%%%%%%%%%%%%%
   \subsection{ Exemplo para consequência lógica local.}%%%%%%%%%%%%%%%%%%%%%%%%%%%%%%%%%%%%%%%%%%%%%%%%%%%%%%%%%%%%%%%%%%%%%%%%%%%%%%
   \begin{quote}
   $ \{ \square (\varphi \rightarrow \varphi'), \Diamond \varphi \} \Vdash_{l}  \Diamond \varphi'$.
  \end{quote}
  Prova. Se M,s $\models \square (\varphi \rightarrow \varphi' )$ e M,s $ \models \Diamond \varphi $ então M,s $\models \Diamond \varphi $ para todo s $\in $ S.\\
  Por contradição, suponha que M,s $\nvDash \Diamond \varphi'$. Isso implica que: Qualquer que seja t, tal que R(s,t), t $\in$ S, M,t $\nvDash \varphi' $. Daí temos que:\\
  M,s $\models \square (\varphi \rightarrow \varphi ')$ e M,s $\models \Diamond \varphi $ por hipótese.\\
  Logo existe t, tal que M,t $\models \varphi $, R(s,t) para que M,s $\models \Diamond \varphi$. E como para todo t, tal que R(s,t), M,t $\nvDash \varphi '$, logo M,s $\nvDash \square (\varphi \rightarrow \varphi')$. Contradição! Logo se M,s $\models \square (\varphi \rightarrow \varphi' )$ e M,s $ \models \Diamond \varphi $ então M,s $\models \Diamond \varphi $ para todo s $\in $ S.
  %%%%%%%%%%%%%%%%%%%%%%%%%%%%%%%%%%%%%%%%%%%%%%%%%%%%%%%%%%%%%%%%%%%%%%%%%%%%%%%%%%%%%%%%%%%%%%%%%%%%%%%%%%%%%%%%%%%%%%%%%%%%%%%%%%%
  %%%%%%%%%%%%%%%%%%%%%%%%%%%%%%%%%%%%%%%%%%%%%%%%%%%%FIM DO CÁPITULO %%%%%%%%%%%%%%%%%%%%%%%%%%%%%%%%%%%%%%%%%%%%%%%%%%%%%%%%%%%%%%%
  %%%%%%%%%%%%%%%%%%%%%%%%%%%%%%%%%%%%%%%%%%%%%%%%%%%%%%%%%%%%%%%%%%%%%%%%%%%%%%%%%%%%%%%%%%%%%%%%%%%%%%%%%%%%%%%%%%%%%%%%%%%%%%%%%%%
   
%%%%%%%%%%%%%%%%%%%%%%%%%%%%%%%%%%%%%%%%%%%%%%%%%%%%%%%%%Novo capitulo%%%%%%%%%%%%%%%%%%%%%%%%%%%%%%%%%%%%%%%%%%%%%%%%%%%%%%%%%%%%%
\chapter{Sistemas Axiomáticos Modais.}   
\hspace{0.5cm} Para apresentar sistemas axiomáticos modais, preciso definir conceitos para a definição do sistema axiomático K, que é o mais básico dos sistemas aximáticos e que representa a interseção de todos os outros sistemas axiomáticos. 
\subsection{Derivação.}%%%%%%%%%%%%%%%%%%%%%%%%%%%%%%%%%%%%%%%%%%%%%%%%%%%%%%%%%%%%%%%%%%%%%%%%%%%%%%%%%%%%%%%%%%%%%%%%%%%%%%%%%%%%
\hspace{0.5cm}Sejam $\vdash$ o símbolo da derivação, $\Psi$ um conjunto de fórmulas tal que $\Psi\subseteq$ FormM e $\varphi_{1},...,\varphi_{n} \in \Psi$, se $\Psi \vdash \varphi$( psi deriva phi) então $\varphi$ é um axioma, ou $\varphi \in \Psi$ ou $\chi \in \Psi$, $\chi \rightarrow \varphi \in \Psi$ e $\varphi$ é obtida por meio da regra de inferência modus ponens.
%%%%%%%%%%%%%%%%%%%%%%%%%%%%%%%%%%%%%%%%%%%%%%%%%%%%%%%%%%%%%%%%%%%%%%%%%%%%%%%%%%%%%%%%%%%%%%%%%%%%%%%%%%%%%%%%%%%%%%%%%%%%%%%%%%%%
\subsection{Fórmulas modais atômicas.}%%%%%%%%%%%%%%%%%%%%%%%%%%%%%%%%%%%%%%%%%%%%%%%%%%%%%%%%%%%%%%%%%%%%%%%%%%%%%%%%%%%%%%%%%%%%%
\hspace{0.5cm} Seja PA o conjunto das fórmulas proposicionalmente atômicas, PA é definido por:\\
PA = P $\cup$ \{$\square \varphi $ : $\varphi \in $FormM\} $\cup $ \{$\Diamond \varphi $: $\varphi \in $ FormM \}\\
\indent Os elementos do conjunto PA podem ser vistos como um conjunto de símbolos proposicionais e sendo assim, fazem parte da linguagem proposicional. Desse modo, podemos substituir por seus elementos, variáveis dos axiomas da lógica proposicional, a exemplo de $\square \varphi \rightarrow (\psi \rightarrow \square \varphi )$ que é uma instancia do axioma $ \varphi \rightarrow (\psi \rightarrow  \varphi )$.
\subsection {Regras de inferência.}%%%%%%%%%%%%%%%%%%%%%%%%%%%%%%%%%%%%%%%%%%%%%%%%%%%%%%%%%%%%%%%%%%%%%%%%%%%%%%%%%%%%%%%%%%%%%%%%
\hspace{0.5cm} Seja $\Phi\subseteq $ FormM, $\Phi$ é fechado para o modus ponens se:\\
$\varphi \rightarrow \varphi ' \in \Phi $ e $\varphi \in \Phi $, então $\varphi '\in \Phi $.\\
\indent O conjunto $\Phi \subseteq $ FormM é fechado para a regra de inferência necessitação(do inglês necessitation):\\
Se $\varphi \in \Phi $ então $\square \varphi \in  \Phi $.
\subsection{Sistema K.}
\hspace{0.5cm} O sistema axiomático K é definido por:\\
K1: Todos os axiomas da lógica proposicional.\\
K2: $\square (\varphi \rightarrow \psi )\rightarrow (\square \varphi \rightarrow \square \psi )$. \\
\indent O sistema K é fechado para as regras de necessitação e modus ponens, podendo derivar ou instanciar fórmulas através dos axiomas da lógica proposicional.
\subsection{Sistema T.}%%%%%%%%%%%%%%%%%%%%%%%%%%%%%%%%%%%%%%%%%%%%%%%%%%%%%%%%%%%%%%%%%%%%%%%%%%%%%%%%%%%%%%%%%%%%%%%%%%%%%%%%%%%%%
\hspace{0.5cm} O sistema T é uma extensão do sistema K com a adição do axioma:\\
T1: $\square \varphi \rightarrow \varphi $.\\
o que representa a classe dos enquadramentos reflexivos, ou seja, o axioma é válido somente para enquadramentos cuja relação possui a propriedade reflexiva.\\
Prova. Por contradição, suponha que $\square \varphi \rightarrow \varphi $ não é válido, ou seja, existe um modelo de kripke M, e um estado s, s$ \subseteq $ S, S $\subseteq $ E, E = ( S,R), R com propriedade reflexiva, tal que M,s $\models \square \varphi $ e M,s $\nvDash \varphi $. Para que M,s $\models \square \varphi $, então para todo s', tal que R(s,s'), M,s'$\models \varphi $. Mas como a relação R é reflexiva, então s = s', logo M,s $\models \varphi $ para todo s, o que é uma contradição pois, por hipótese, M,s $\nvDash \varphi $. De outra mão, o axioma T1 não é válido para modelos de kripke cujo enquadramento não possua relação reflexiva. Suponha um modelo de kripke M' = (E',V'), onde E' não possui relacionamento reflexivo. Seja um estado s, s $\in $ S' ta que M',s'$\nvDash \varphi $ e para todo t, tal que R(s,t), M',t $\models \varphi $. Logo tem-se que M',s$\models \square \varphi $ e M',s $\nvDash \varphi $. O que mostra que $\square \varphi \rightarrow \varphi $ não é válido.
\subsection{Sistema S4.}
\hspace{0.5cm} O sistema S4 é composto pelo axioma:\\
S4: $\square \varphi \rightarrow \square \square \varphi  $ \\
e pelo sistema T. Se uma fórmula é necessariamente verdadeira, então necessariamente, ela é necessariamente verdadeira. Esse axioma caracteriza a classe dos enquadramentos transitivos. Agora vou mostrar que para qualquer que seja o modelo de kripke cujo enquadramento possui a relação de propriedade transitiva, o axioma S4 é válido.\\
Prova.Sejam os estados s,s',s'' $\in$ S, S $\subseteq $ M, M qualquer. Sejam os seguintes relacionamentos: R(s,s') , R(s',s'') e pela propriedade da transitividade, R(s,s''). Por contradição, suponha que $\square \varphi \rightarrow \square \square \varphi  $ não é válida, logo existe a seguinte configuração: M,s $\models \square \varphi $ e M,s $\nvDash \square \square \varphi $. Se M,s $\models \square \varphi $, então para todo s' tal que R(s,s', M,s'$\models \varphi $. E pela suposição contraditória, M,s $\nvDash \square \square \varphi $, então M,s''$\nvDash \varphi $. Mas aí, temos a contradição, pois se M,s''$\nvDash \varphi $, então M,s $\nvDash \square \varphi $. Veja que por outro lado, esse axioma não é valido em modelos cujos enquadramentos não possuem a propriedade transitiva. Sejam s,s',s'' $\in$ S, S $\subseteq $ E, E um enquadramento que não possui propriedade transitiva. Sejam as seguintes relações: R(s,s') e R(s',s''). Sejam as seguintes interpretações: M,s $\models \square \varphi $ e M,s $\nvDash \square \square \varphi $. Daí temos que para todo s', tal que R(s,s'), M,s' $\models \varphi $ e que M,s'' $\nvDash  \varphi $. Logo, fica claro que M,s $\nvDash \square \square \varphi $, e que portanto, o axioma não é válido para enquadramentos que não sejam transitivos.
\subsection{ Sistema S5.}
\hspace{0.5cm} O sistema axiomático S5 é composto pelo sistema S4 e pelo axioma:\\
S5: $\Diamond \square \varphi \rightarrow \varphi $, \\
e caracteriza a classe dos enquadramentos simétricos. Prova. Seja um modelo de Kripke M qualquer, tal que o enquadramento E possui  propriedade da simetria no relacionamento entre os estados. Sejam os estados s,s',s'' e os relacionamentos R(s,s') e R(s',s''). Por contradição, suponha que $\Diamond \square \varphi \rightarrow \varphi $ não é válido, ou seja, existe M,s $\models \Diamond\square  \varphi $ e M,s $\nvDash \varphi $. Mas aí temos que M,s'$\models \varphi $ para todo s' R(s,s') e M,s'' $\models \varphi $. Mas pela propriedade da simetria, existe a relação R(s'',s'), o que nos leva a uma contradição pois, se M,s' $\nvDash \square \varphi $ então M,s $\nvDash  \Diamond \square \varphi $.
\indent É fácil verificar que o axioma não é válido em um enquadramento que não possui a propriedade simétrica. Usando o mesmo exemplo, se não há relacionamento simétrico entre os estados, principalmente por s, s', M,s $\models \Diamond \square \varphi $ e M,s$\nvDash \varphi $, mostrando que o axioma S5 não é valido para enquadramentos cuja propriedade simétrica no relacionamento não existe.


   
   

  
\end{document}
