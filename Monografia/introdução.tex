\section{Introdução.}
     \hspace{0.5cm} Nesta seção apresentarei a lógica modal baseada na lógica proposicional. Mas primeiramente, o que é lógica modal? Lógica modal é o estudo das proposições modais e as relações lógicas entre elas. Proposições modais são proposições que enunciam a maneira com que o predicado convém ao sujeito. Exemplos de proposições modais:\\
	\indent É possível que irá chover amanhã.\\
	\indent É possível para seres humanos viajarem até marte.\\
	\indent Necessariamente está(é necessário que esteja) chovendo aqui agora ou não.\\
       \indent Usarei aqui os dois operadores modais necessidade e possibilidade. Quando uma asserção usa o operador da necessidade, ela é necessariamente verdadeira, ou seja, não existe um mundo, situação ou estado em que ela seja falsa. Por outro lado , uma asserção pode ser possível, ou seja, existe pelo menos um mundo,  situação ou estado em que a asserção é verdadeira e podendo haver outras em que ela seja falsa. \\
       \indent  O operador modal da necessidade aqui será o $\Box$, logo sendo uma fórmula $\varphi$, $\Box$$\varphi$ é uma nova fórmula que se lê $\varphi$ necessário, cuja fórmula é necessariamente verdadeira. O operador modal para possibilidade é o $\Diamond$, daí seguindo o mesmo raciocínio, temos $\Diamond$$\varphi$ como uma nova fórmula tal que a asserção possui leitura $\varphi$ possível.\\
     \indent A lógica modal serve como base para outras lógicas, e consequentemente, possui interpretações diferentes para os símbolos modais já determinados,  ou como em alguns casos, um acréscimo de seus operadores modais. Cito aqui alguns exemplos para lógicas com base na lógica modal:\\
         \indent Lógica temporal, que é uma lógica que representa a veracidade das proposições com o passar do tempo, ou seja, uma asserção pode ser sempre verdadeira no futuro, ou alguma vez no futuro. A saber, temos o operador $\Box$, como o representante de “sempre no futuro” e o operador $\Diamond$ como o operador de “alguma vez no futuro” \\
       \indent Lógica doêntica, que é a lógica conhecida como lógica das obrigações e permissões. Neste caso, temos o operador modal $\Box$ representando uma asserção obrigatória, e temos o operador $\Diamond$ representando a asserção que é permitida.\\
          \indent A lógica epistêmica, também conhecida como a lógica do conhecimento, é uma lógica em que os símbolos modais são interpretados da seguinte forma: $\Box$ representa a asserção que é conhecida, ou seja, um agente i sabe $\varphi$, que tem expressão na forma de $\Box$$_{i}$$\varphi$. Similarmente, um agente i acredita  em $\varphi$, cuja expressão equivale a $\Diamond$$_{i}$$\varphi$. Os símbolos modais  $\Box$ e $\Diamond$ costumeiramente são substituídos por K e B respectivamente. K vem do inglês Knowledge (conhecimento) e B de Belief (crença,acreditar).\\
